 % PACKAGES
\documentclass[a4paper, hidelinks, 12pt]{report}
\usepackage[margin=1in]{geometry}
\usepackage{amsfonts,amsmath,amssymb}
\usepackage[none]{hyphenat}
\usepackage{fancyhdr}
\usepackage{graphicx}
\usepackage{float}
\usepackage[nottoc,notlot,notlof]{tocbibind}
\usepackage{hyperref}
\usepackage{longtable}
\usepackage[utf8]{inputenc}
\usepackage{booktabs}
\usepackage{multirow}
\usepackage{booktabs}
\usepackage[font=footnotesize]{caption}
\usepackage[flushleft]{threeparttable}
\usepackage{amsmath}
\usepackage{relsize}
\usepackage[super,negative]{nth}
\usepackage{enumerate}
\usepackage{float}
\usepackage{rotating}
\usepackage[dvipsnames]{xcolor}
\usepackage{listings}
\usepackage{float}
\restylefloat{table}

%%%%%%%%%%%%

% DOC STYLES
\makeatletter
\def\thickhrulefill{\leavevmode \leaders \hrule height 1ex \hfill \kern \z@}
\def\@makechapterhead#1{
	\vspace*{4\p@}
	{\parindent \z@ \centering \reset@font
		\thickhrulefill\quad
		\scshape \@chapapp{} \thechapter
		\quad \thickhrulefill
		\par\nobreak
		\vspace*{4\p@}
		\interlinepenalty\@M
		\hrule
		\vspace*{4\p@}
		\Huge \bfseries #1\par\nobreak
		\par
		\vspace*{4\p@}
		\hrule
		\vskip 50\p@
}}
\def\@makeschapterhead#1{
	\vspace*{4\p@}
	{\parindent \z@ \centering \reset@font
		\thickhrulefill
		\par\nobreak
		\vspace*{4\p@}
		\interlinepenalty\@M
		\hrule
		\vspace*{4\p@}
		\Huge \bfseries #1\par\nobreak
		\par
		\vspace*{4\p@}
		\hrule
		\vskip 50\p@
}}

\pagestyle{fancy}
\fancyhead{}
\fancyfoot{}
\fancyhead[L]{\slshape\MakeUppercase{\textbf{ATD}}}
\fancyhead[R]{\slshape{Avila, Schiatti, Virdi}}
\fancyfoot[C]{\thepage}
\renewcommand{\footrulewidth}{1pt}
\renewcommand{\headrulewidth}{1pt}
\linespread{1.3}
%\floatstyle{boxed}
\restylefloat{figure}
\setlength\parindent{0pt}


\usepackage{listings}
\usepackage{color}
 
\definecolor{codewhite}{rgb}{0.95,0.95,0.92}
\definecolor{codeblack}{rgb}{0,0,0}
 
\lstdefinestyle{mystyle}{
    backgroundcolor=\color{codewhite}, 
    keepspaces=false,
    tabsize=1,
    basicstyle=\ttfamily\small
}
 
\lstset{style=mystyle}

%%%%%%%%%%%%

% COMMANDS
\newcommand\requirement[1]{\item[{[REQ-#1]}] }
\newcommand\test[1]{\item[{[TEST-#1]}] }
%%%%%%%%%%%%

%BODY
\begin{document}
	\begin{titlepage}
		\centering
		\vspace*{0.7 cm}
		\includegraphics[scale = 0.85]{../Assets/PolimiLogo.png}\\[1.6 cm]
		\textsc{\large Department of Computer Science and Engineering}\\[1.8 cm]

		\rule{\linewidth}{0.2 mm} \\[0.4 cm]
		{ \huge \bfseries Acceptance Test Deliverable (ATD)}\\
		\rule{\linewidth}{0.2 mm} \\[1.5 cm]

		\textsc{\Large TrackMe}\\[3 cm]

		\begin{minipage}{0.4\textwidth}
			\begin{flushleft} \large
				\emph{Authors:}\\
				\textbf{Avila}, Diego \\
				\textbf{Schiatti}, Laura \\
				\textbf{Virdi}, Sukhpreet
			\end{flushleft}
		\end{minipage}~
		\begin{minipage}{0.4\textwidth}
			\begin{flushright} \large
				903988 \\
				904738 \\
				904204
			\end{flushright}
		\end{minipage}\\[2 cm]


		{\large January \nth{20} , 2019}\\[2 cm]

		\vfill
	\end{titlepage}

	\pagenumbering{roman}
	\tableofcontents
%	\thispagestyle{empty}
	\newpage
%	\listoffigures
	\listoffigures
%	\thispagestyle{empty}
	\clearpage
	\pagenumbering{arabic}
	\setcounter{page}{1}

	\chapter{Introduction}
	\section{Purpose}
	This document is the Acceptance Test Plan (ATP) for \textbf{TrackMe} software. The acceptance test verifies that the system works as required and validates that the correct functionality has been delivered.  The main purpose of this test is to evaluate the system's compliance with the requirements and verify if it is has met the required criteria for delivery to end users. 
	
	\section{Analyzed project}
	Documents and source code regarding the project that was analyzed can be consulted from the repository 
		\href{https://github.com/DavideRutigliano/FerranteMattaRutigliano}{\textbf{https://github.com/DavideRutigliano/FerranteMattaRutigliano}}.\\
		
	\textbf{Authors: } 
	\begin{itemize}
		\item Davide Rutigliano
		\item Davide Matta
		\item Claudio Ferrante
	\end{itemize}
	
	\subsection{Abbreviations}
	\begin{itemize}
		\item $[REQ-n]$: n-functional requirement.
		\item $[TEST-n]$: n-test case.
	\end{itemize}

\section{Overview}
This document describes the focuses of the validation activity and the results obtained. The below table describes how the activities under Acceptance Testing was formulated:

\begin{table}[H]
  \centering
  \caption{Acceptance Testing Schedule}
    \begin{tabular}{|l|l|}
    \toprule
   \multicolumn{1}{|c|}{\multirow{3}[2]{*}{\textit{\textbf{Activity}}}} &  \\
& \multicolumn{1}{c|}{\textit{\textbf{Checkpoint}}} \\
 &  \\
    \midrule
    \textit{Plan Acceptance Testing for  TrackMe} & \textit{Preliminary Acceptance Test Scenario} \\
    \midrule
    \textit{Identify Test Materials} & \textit{Preliminary Acceptance Test Data} \\
    \midrule
    \textit{Establish Acceptance Test Environment} & \textit{Acceptance Test Environment Inventory} \\
    \midrule
    \multirow{2}[2]{*}{\textit{Conduct Acceptance Test Readiness Review}} & \textit{Draft Acceptance Test Plan} \\
          & \textit{Completed Test Readiness Checklist} \\
    \midrule
    \textit{Execute Tests} & \textit{Acceptance Test Progress} \\
    \midrule
    \textit{Complete Acceptance Testing} & \textit{Acceptance Test Summary } \\
    \midrule
    \textit{Document Acceptance Testing} & \textit{Final Acceptance Test Document} \\
    \bottomrule
    \end{tabular}%
  \label{tab:addlabel}%
\end{table}%
Test traceability is used to verify that functional requirements are accounted for, and tested in the delivered software.  It provides traceability between the requirements and the test materials.\\

Regarding the preparation of environment preparatory activities involve constructing the actual acceptance test environment and installing the hardware and software, including the software to be tested, in order to ensure the configuration is correct.  Changes need to be closely managed and controlled. Prior to an installation or configuration, the acceptance test team reviews a list of affected components and the planned installation or configuration activities.  Deviations from the planned activities are recorded and reported to the acceptance test team.  The acceptance test team begins acceptance testing by conducting an initial series of ad hoc, diagnostic tests designed to exercise the acceptance test environment and verify that major system software capabilities are available and functioning.\\\\
With the Acceptance Test Plan as the guiding document, components reviewed include, but are not limited to:
\begin{itemize}
\item{}Software components
\item{}Database
\item{}Environmental components
\item{}Documentation
\item{}Known problems and outstanding issues from the System Test
\item{}Inventory of the acceptance test environment

\end{itemize}
If the status of one or more of the components is unsatisfactory, the parties identify corrective actions and schedule another ATD.  If the status is acceptable, then testing will proceed.\\

Acceptance test execution is an iterative process that begins with the initial execution of the planned tests.  If no defects are discovered, then the test procedure has “passed.” If defects are discovered, they are documented.

	\chapter{Installation setup}
	%What you did to install the prototype as well as any problem you have faced in this phase, any incoherency you may have found when following the documentation accompanying the software.	

	The application was tested by means of a Linux machine, following the steps provided in the \textit{Implementation and Testing Deliverable}. 
	
	\section{Server Build and Installation}
	
\begin{itemize}
	\item \textbf{Database}
		\begin{enumerate}
			\item Install PostgreSQL Database and pgAdmin
			\begin{lstlisting}[language=bash]
				$ sudo apt install postgresql
				$ sudo -u postgres psql postgres
				$ \password postgres
				$ sudo apt install pgadmin4
			\end{lstlisting}
			
			\item Launch pgAdmin 
			\begin{lstlisting}[language=bash]
				$ sudo pgadmin4
			\end{lstlisting}
			
			\begin{enumerate}
				\item Select pgAdmin 4 $>$ New pgAdmin 4 window
				\item Create and setup a new server  [Fig. \ref{fig:create_server}].
			
				\begin{figure}[H]
					\centering
				\includegraphics[width=0.4\textwidth]{images/create_server.jpeg}
					\caption[Data4Help Create Server]{Data4Help Create Server}
				\label{fig:create_server}
			\end{figure}
				
				\item Press Save.
			\end{enumerate}			
			
			\item Create and setup a new user [Fig. \ref{fig:privileges}].
			
			\begin{enumerate}
				\item Right click on Login/Group Roles $>$ create $>$ Login/Group Role
				\item In the tab \textit{General}, set name = trackmeadmin.
				\item In the tab \textit{Definition}, set password = password.
				\item In the tab \textit{Privileges}, set Superuser = yes and Can login = yes.
			\end{enumerate}
			
			\begin{figure}[H]
					\centering
				\includegraphics[width=0.5\textwidth]{images/priviledges.jpeg}
					\caption[Data4Help Setup User Privileges]{Data4Help Setup User privileges}
				\label{fig:privileges}
			\end{figure}
			
			\item  Create a new database
			\begin{enumerate}
				\item Right click in \textit{Databases} $>$ Create $>$ Database
				\item In the tab \textit{General}, set name = trackme\_db.
			\end{enumerate}						
		\end{enumerate}
		
	\item \textbf{Build with Maven}
		\begin{lstlisting}[language=bash]
			$ cd FerranteMattaRutigliano/DeliveryFolder/software/server
			$ chmod +x mvnw
			$ ./mvnw package
		\end{lstlisting}
\end{itemize}
	
	\section{Client Build and Installation}
	
	\begin{itemize}
		\item  Build with Android Studio 
	
		\begin{enumerate}
			\item Download and install  \href{https://developer.android.com/studio/}{\textbf{Android Studio}}
			
			\begin{figure}[H]
					\centering
				\includegraphics[width=0.5\textwidth]{images/android_studio.png}
					\caption[Android Studio]{Android Studio}
				\label{fig:android_studio}
			\end{figure}
			
			\item Select \textit{Open an existing android studio project} 
			\item Open ”ferrantemattarutigliano/software/client/build.gradle”.
			
			\item Configure the project. Go to  \textit{Select file} $>$ project structure. 
			
			\begin{figure}[H]
					\centering
				\includegraphics[width=0.5\textwidth]{images/project_structure.png}
					\caption[Configure Project Structure]{ Configure Android Studio}
				\label{fig:project_structure}
			\end{figure}
			
			\item Install \textbf{Android NDK} by clicking in the \textit{Download} link (Fig. \ref{fig:android_studio}).
			
			\begin{figure}[H]
					\centering
				\includegraphics[width=0.5\textwidth]{images/android_NDK.png}
					\caption[Install Android NDK]{Install Android NDK}
				\label{fig:android_NDK}
			\end{figure}
			
			\item Run the application and select a deployment target. For testing a Samsung J8 was used instead of a virtual device \ref{fig:deployment_target}.  
			
			\begin{figure}[H]
					\centering
				\includegraphics[width=0.5\textwidth]{images/deployment_target.png}
					\caption[Choose Deployment Target]{Choose Deployment Target}
				\label{fig:deployment_target}
			\end{figure}
			
			\item Download the corresponding SDK.
			
			\begin{figure}[H]
					\centering
				\includegraphics[width=0.5\textwidth]{images/install_sdk.png}
					\caption[Install SDK]{Install SDK}
				\label{fig:install_sdk}
			\end{figure}

	\section{External Device App Build and Installation}
	It was possible to run the device emulator. But it was not possible to run both emulators at the same time and make them communicate. The main problem was due to the next command, it was not working
	
		\begin{lstlisting}[language=bash]
			$ adb -d forward tcp :5601 tcp :5601
		\end{lstlisting}
	
		\end{enumerate}
	\end{itemize}

	\chapter{Acceptance test cases}
	%The acceptance test cases you have applied and the corresponding outcome: you can extract test cases by analysing the RASD the other team has developed, the list of features that the team has actually developed as well as any other case that you, as a user of the application, think is appropriate (please provide motivation for your choice). 
	
	\section{Motivation}
	The system requirements are categorized according to their importance regarding the scope of the problem and the main functionalities to be implemented in order to have a minimum viable prototype as \textbf{HIGH}, \textbf{MEDIUM} and \textbf{LOW}. \\

	All requirements defined as \textbf{MEDIUM} and \textbf{HIGH} were tested and their outcomes. Even if \textit{registration} (for both third parties and individuals) was not consider relevant, it was also assessed in order to have available users for testing.

	\section{Tested features}
	As commented in the ITD document, sometimes when trying to sign-up or log-in a user, or when doing generic requests, the response was a \textit{Connection timeout} error message. Since this was mentioned in the ITD document, every time we got the message it was not considered as an issue.\\
	
	On the other hand, the template for defining the tests is the following:\\
	\textbf{Input}: The inputs used to test the requirement.\\
	\textbf{Output}: The obtained output after following the steps in the related use cases.\\
	\textbf{Result}: Passed of Fail. The result of the procedure.\\
	\textbf{Possible bugs}: If there are possible bugs in the procedure, they are listed here.\\
	\textbf{Comments}: Some relevant comments about the procedure.

	\subsection{Data4Help}
	\begin{itemize}
		\requirement{1} A customer not signed-on must be able to begin the Individual’s registration process to TrackMe providing a username, a password and its organization data.  \textbf{[LOW]}
		
	\begin{itemize}
		\test{1}: Register an individual for the first time [Fig. \ref{fig:register_individual}].
			\begin{itemize}
			\item \textbf{Inputs: } username=Individuo, ssn=344918500			
			\item \textbf{Outcome: } OK. Individual registered successfully. 
			\item \textbf{Result: } Passed. 
			\end{itemize}
			
		\begin{figure}[H]
					\centering
				\includegraphics[width=0.2\textwidth]{images/register_individual.jpeg}
					\caption[Data4Help Register Individual]{Data4Help Register Individual}
				\label{fig:register_individual}
			\end{figure}
			
		\test{2}: Register an individual with an already registered ssn.
			\begin{itemize}
			\item \textbf{Inputs: } username=individual2, ssn=344918500
			\item \textbf{Outcome: } Error. Since ssn is duplicated, the registration was not allowed.
			\item \textbf{Result: } Passed. 
			 \end{itemize}	
			 
		\item \textbf{Possible bugs}: Birth date field is not validated, a user with a future birth date can be registered.
	\end{itemize}
	
		\requirement{2} A customer not signed-on must be able to begin the Third Party’s registration process to TrackMe providing a username, a password and its organization data. \textbf{[LOW]}
		\begin{itemize}
		\test{3}: Register third party for the first time [Fig. \ref{fig:register_third_party}].
			\begin{itemize}
			\item \textbf{Inputs: } username=Tp1			
			\item \textbf{Outcome: } Ok. Third party was registered.
			\item \textbf{Result: } Passed. 
			\end{itemize}
			
		\begin{figure}[H]
					\centering
				\includegraphics[width=0.2\textwidth]{images/register_third_party.jpeg}
					\caption[Data4Help Register Third Party]{Data4Help Register Third Party}
				\label{fig:register_third_party}
			\end{figure}
			
		\test{4}: Register a Third party with an already registered vat.
			\begin{itemize}
			\item \textbf{Inputs: } username=Tp1
			\item \textbf{Outcome: } Error. The vat was duplicated, registration not allowed.
			\item \textbf{Result: } Passed. 
			 \end{itemize}	
	\end{itemize}
	
\requirement{3}The system must provide a log-in interface for already registered users, not previously signed into the application. \textbf{[MEDIUM]}
		\begin{itemize}
		\test{5}: Login user with valid credentials [Fig. \ref{fig:register_third_party}].
			\begin{itemize}
			\item \textbf{Inputs: }Password=1234567890, Username=Tp1			
			\item \textbf{Outcome: } OK. User Logged in successfully.
			\item \textbf{Result: } Passed. 
			\end{itemize}
			
			
		\test{6}: Login with invalid credentials
			\begin{itemize}
			\item \textbf{Inputs: } Username=Tp1, Password=12341231
			\item \textbf{Outcome: } Login Error.
			\item \textbf{Result: } Passed. 
			 \end{itemize}	
	\end{itemize}
	
		\requirement{4}TrackMe must provide to users the possibility to change their username. \textbf{[LOW]}
		
		\requirement{5}TrackMe must provide to registered users the possibility to change their password. \textbf{[LOW]}
		\requirement{6}The system must provide the possibility to the user of logging out. \textbf{[LOW]}
		\requirement{7}The system must provide the possibility to change Individual’s personal data. \textbf{[LOW]}
		\requirement{8}The system must provide the possibility to change Third Party’s organization data. \textbf{[LOW]}
		\requirement{9} Data4help must allow the Third-Parties to send a request to a particular individual, provided his SSN/FC.  \textbf{[HIGH]}
		
				\begin{itemize}
		\test{7}: Send request to individual [Fig. \ref{fig:request_individual}].
			\begin{itemize}
			\item \textbf{Inputs: } SSN 344918500			
			\item \textbf{Outcome: } Ok. Request was sent to individual successfully.
			\item \textbf{Result: } Passed. 
			\end{itemize}
			\item{\textbf{Possible Bugs:}} a third party can send 2 requests to the same user, the user can accept one of them and reject the other, and the DB will be in an inconsistent state.
			
		\begin{figure}[H]
					\centering
				\includegraphics[width=0.2\textwidth]{images/request_individual.jpeg}
					\caption[Data4Help Request Individual]{Data4Help Request Individual}
				\label{fig:request_individual}
			\end{figure}
			
		\test{8} : Send request with an invalid ssn
			\begin{itemize}
			\item \textbf{Inputs: } SSN 1112121a (invalid)
			\item \textbf{Outcome: } Error. Invalid SSN.
			\item \textbf{Result: } Passed. 
			 \end{itemize}	
	\end{itemize}
	
	\requirement{10} Data4help must allow the Third-Parties to generated a request for a group of individuals.  \textbf{[HIGH]}
	
						\begin{itemize}
		\test{9}: Request for group of individuals [Fig. \ref{fig:request_group_individuals}].
			\begin{itemize}
			\item \textbf{Inputs: } City=Milan	
			\item \textbf{Outcome: } The system creates the request.
			\item \textbf{Result: } Passed.
			\item \textbf{Comment: } Since the database was empty, and the anonymization constraint was that more than 1000 users should fulfil the filter, we removed the constraint and tested the requirement. Moreover, we used an script to create 1000 users provided by the team, and the anonymization constraint is working ok.  
			\end{itemize}
			\begin{figure}[H]
					\centering
				\includegraphics[width=0.2\textwidth]{images/request_group_individuals.jpeg}
					\caption[Data4Help Request Group Individual]{Data4Help Request Group Individual}
				\label{fig:request_group_individuals}
			\end{figure}

	\end{itemize}
	
	\requirement{11} The system must be able to anonymize users’ requested data. \textbf{[HIGH]}
	\begin{itemize}
		\test{10}: Anonymize data.
			\begin{itemize}
			\item \textbf{Inputs: } -			
			\item \textbf{Outcome: } cannot be tested because there is no enough data.
			\item \textbf{Result: } Testing was not possible. 
			\end{itemize}	
	\end{itemize}
	
	\requirement{12} Data4help must allow the Third-Parties to choose if subscribe to new data associated with a particular Individual. \textbf{[MEDIUM]}
			
				\begin{itemize}
		\test{11}: Subscribe to new data of an individual while requesting for data [Fig. \ref{fig:data_subscription}].
			\begin{itemize}
			\item \textbf{Inputs: } SSN 344918500			
			\item \textbf{Outcome: } If the request is accepted by the individual, the subscription is created automatically.
			\item \textbf{Result: } Passed.
			\item \textbf{Possible bug: } Even if the third party is not subscribed to an Individual, it receives the data.
			\end{itemize}
						
		\begin{figure}[H]
					\centering
				\includegraphics[width=0.5\textwidth]{images/data_subscription.jpeg}
					\caption[Data4Help data subscription]{Data4Help data subscription}
				\label{fig:data_subscription}
			\end{figure}		
		
	\end{itemize}
	
	\requirement{13} The system must be able to periodically query the database in order to get new data as soon as they are produced. \textbf{[MEDIUM]}	
			
				\begin{itemize}
		\test{12}: Run the query for data periodically.
			\begin{itemize}
			\item \textbf{Inputs: } -			
			\item \textbf{Outcome: } None of the provided documents state the meaning of periodically and there is no way of testing it.

			\item \textbf{Result: } Testing was not possible. 
			\end{itemize}
			
	\end{itemize}

	\requirement{14} Data4help must allow the Third Party to see a list of sent requests and related Individual’s data. \textbf{[LOW]}
	\requirement{15} Data4help must allow the Individual to see a list of the received requests from Third Parties. \textbf{[LOW]}
	\requirement{16} Data4help must allow the Individual to accept or reject a request from the list. \textbf{[MEDIUM]}
			
				\begin{itemize}
		\test{13}: Accept or Reject the request [Fig. \ref{fig:request_acceptreject}].
			\begin{itemize}
			\item \textbf{Inputs: } password=1234567890, username=Tp1			
			\item \textbf{Outcome: } Ok. The status is updated and the request is removed from the list (no update options).
			\item \textbf{Result: } Passed. 
			\end{itemize}		
			
		\begin{figure}[H]
					\centering
				\includegraphics[width=0.2\textwidth]{images/manage_requests.jpeg}
					\caption[Data4Help Accept/Reject Request]{Data4Help Accept/Reject Request}
				\label{fig:request_acceptreject}
			\end{figure}

	\end{itemize}
	
	\requirement{17} Data4help must allow the Individual to connect an external device through BT or NFC. \textbf{[HIGH]}
	\begin{itemize}
		\item \textbf{Comment: } We couldn't test this requirements due to some issues when trying to connect to the external device.
	\end{itemize}
	
	
	\end{itemize}
	\subsection{Track4Run}
	
	\begin{itemize}
	\item \textbf{Runs management}
	\requirement{24} Track4Run must allow the organizer to create a run with a date, time, duration and a path. \textbf{[HIGH]}
					\begin{itemize}
		\test{14}: Create a run with valid data set [Fig. \ref{fig:run_path}].
			\begin{itemize}
			\item \textbf{Inputs: } Set title, Date, time and path.			
			\item \textbf{Outcome: } Ok. Path was created successfully.
			\item \textbf{Result: } Passed. 
			\end{itemize}			
			
		\begin{figure}[H]
					\centering
				\includegraphics[width=0.2\textwidth]{images/define_path.jpeg}
					\caption[Track4Run Create path]{Track4Run Create path}
				\label{fig:run_path}
			\end{figure}
			
		\test{15} : Create a run with invalid fields
			\begin{itemize}
			\item \textbf{Inputs: }Select only 2 points
			\item \textbf{Outcome: }Path with less than 2 points are not valid. All fields are validated.
			\item \textbf{Result: } Passed. 
			 \end{itemize}	
	\end{itemize}
	\requirement{25} The system must provide an interface that allows the user to define a path on an interactive map. \textbf{[MEDIUM]}
						\begin{itemize}
		\test{16}: Set path on an interactive map [Fig. \ref{fig:run_path}].
			\begin{itemize}
			\item \textbf{Result: } Passed (same as above). 
			\end{itemize}
	
	\end{itemize}
	\requirement{26} Track4Run must allow the organizer to start a run previously created. \textbf{[MEDIUM]}
	\begin{itemize}
	\test{17} : Start a run
				\begin{itemize}
			\item \textbf{Inputs: } Select a run to start from the active run list
			\item \textbf{Outcome: } the organizer can check the list of runs he created and start any of them at any time (the system does not check whether it is about to take place or not)
			\item \textbf{Result: } Fail. 
			\end{itemize}
\item{\textbf{Comment:}} the only way of "stopping" a run is deleting it.

\test{18} : Start a run with less than 2 athletes
				\begin{itemize}
			\item \textbf{Inputs: } Select a run in which there are less than 2 athletes participating
			\item \textbf{Outcome: } it was possible, considering domain assumption 18 (A -18).
			\item \textbf{Result: } Passed. 
			\end{itemize}

\test{19} : Start a run that start in future dates
				\begin{itemize}
			\item \textbf{Inputs: } Select run whose start data is in future. 
			\item \textbf{Outcome: } Possible to start the run, but cannot be "watched" by the spectators.
			\item \textbf{Result: } Fail. 
			\end{itemize}

	\end{itemize}


	\requirement{27} Track4Run must allow the organizer to delete a run previously created. \textbf{[LOW]}
	\requirement{28} Track4Run must allow the Athlete to enroll to an already existing run. \textbf{[HIGH]}
		\begin{itemize}
	\test{20} : Enroll a run [Fig. \ref{fig:run_list}].
				\begin{itemize}
			\item \textbf{Inputs: } Select a run to to enroll from active run list
			\item \textbf{Outcome: } Ok. User is allowed to enroll into the run.
			\item \textbf{Result: } Passed.
			\end{itemize}
\begin{figure}[H]
					\centering
				\includegraphics[width=0.5\textwidth]{images/runs_list.jpeg}
					\caption[Track4Run Enroll Run]{Track4Run Enroll Run}
				\label{fig:run_list}
			\end{figure}
\test{21} : Enroll to a run that has already started
				\begin{itemize}
			\item \textbf{Inputs: } Select a run to enroll which has already been started.
			\item \textbf{Outcome: } the user can enroll even if the run has started.
			\item \textbf{Result: } Fail. 
			\end{itemize}

	\end{itemize}
	\requirement{29}  Track4Run must allow the Athlete to unroll a run. \textbf{[LOW]}
 	\requirement{30} Track4Run must allow the Spectator to see the position of the Athletes on a map during a run.\textbf{[HIGH]}
 			\begin{itemize}
	\test{22} : Spectators can see the position of athletes during a run.
				\begin{itemize}
			\item \textbf{Inputs: } press watch, if the run wasn't previously started by the organizer an error message is displayed.
			\item \textbf{Outcome: } We can visualize the map with the path and it is possible to see the position of the athletes.
			\item \textbf{Result: } Pass.
			\end{itemize}

\test{23} : See position of athletes in a a run that hasn't started is not possible.
				\begin{itemize}
			\item \textbf{Inputs: } Click on watch.
			\item \textbf{Outcome: } Since the run has not started, an error message is displayed.
			\item \textbf{Result: } Pass. 
			\end{itemize}

	\end{itemize}
	\end{itemize}

	\chapter{Additional annotations}
	\label{chap:additional_annotations}
	We have some comments regarding the RASD and DD documents. It seems that these documents are outdated since they do not reflect the current status of the application. We think they should be updated in order to represent in a better way the present state of the project.\\
	
	Furthermore, we have seen some not well described requirements, like REQ-13, which states that the system must be able to \textit{periodically} query the database in order to get the data as soon as it is produced. This requirement could not be tested since we don't know what period it should be. Moreover, we think that some assumptions are actually not assumptions at all and should be requirements, like the A-16, which states that \textit{The organizer defines a not empty path} (something which should be impossible). Also, some goals are obvious like GOAL - 01 \textit{Allow a Guest to register as an Individual}. \\
	
	Apart from these comments about the document, we think it is clear enough to understand the system and it's functionality.
	
	\chapter{Conclusion}
	We managed to test all the High and Medium requirements and, even though this application is a prototype and has some bugs, we think it was well developed. In the overall, the most important requirements are working fine.\\
	
	Furthermore, as mentioned in Chapter \ref{chap:additional_annotations}, we think that some of the requirements, goals and assumptions have not been written in a proper way. For example, there are no definitions for \textit{periodically}, so it was impossible to test the rate time the system should check the data base. \\
	
	Finally, we think that the RASD and DD document should be updated to reflect the last changes in the system.\\
	
	To conclude, the Data4Help application is working fine for the Individual or the Third Parties: the individuals can accept or reject the requests, and the third parties receives the notifications every time the information of a user changes; Moreover, the third parties can create group notifications when it satisfies the anonymization constraints. Also, Track4Run section is working fine as well: the organizer can create a run, select the way points and start the run, and the athletes can enrol into a run and the spectators can watch the position of the athletes during the run. Hence, we consider that the application can be \textbf{\emph{accepted}}.

	\chapter{References}
	\begin{itemize}
		\item Requirement Analysis and Specification Document.pdf. Version 1.0 - 10.11.2018.
		\item Design Document.pdf. Version 1.0 - 10.12.2018.
		\item Implementation and Testing Document.pdf. Version 1.0 - 13.01.2019.
	\end{itemize}

\end{document}
