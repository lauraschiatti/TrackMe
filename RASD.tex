\documentclass[12pt]{article}
\usepackage[margin=1in]{geometry}
\usepackage{amsfonts,amsmath,amssymb}
\usepackage[none]{hyphenat}
\usepackage{fancyhdr}
\usepackage{graphicx}
\usepackage{float}
\usepackage[nottoc,notlot,notlof]{tocbibind}
\usepackage{hyperref}
\usepackage{longtable}
\usepackage[utf8]{inputenc}
\usepackage{booktabs}
\usepackage{multirow}
\usepackage{booktabs,caption}
\usepackage[flushleft]{threeparttable}
\usepackage{amsmath}
\usepackage{relsize}
\usepackage[super,negative]{nth}

%\usepackage[demo]{graphicx}

%PAGESTYLE
\pagestyle{fancy}
\fancyhead{}
\fancyfoot{}
\fancyhead[L]{\slshape\MakeUppercase{RASD}}
\fancyhead[R]{\slshape{Avila, Schiatti, Virdi}}
\fancyfoot[C]{\thepage}
\renewcommand{\footrulewidth}{1pt}
\renewcommand{\headrulewidth}{1pt}
\linespread{1.3}

\parindent 0ex
%\renewcommand{baselinestretch}{1.5}


%BODY
\begin{document}
\begin{titlepage}
\begin{center}
\begin{figure}[h]
\includegraphics[scale=1]{Assets/PolimiLogo.png}
\centering
\end{figure}
\centering\textbf{POLITECNICO DI MILANO}\\
\centering\textbf{DEPARTMENT OF COMPUTER SCIENCE ENGINEERING}
\vspace*{4cm}

\line(1,0){450}\\
\Large{\textbf{Requirement Analysis and Specification Document (RASD)}}\\[3mm]
\Large{- TrackMe -}\\
\line(2,0){450}\\
\vfill
By Group: AvilaSchiattiVirdi\\


\today
\end{center}
\end{titlepage}

\tableofcontents
\thispagestyle{empty}
\newpage
\listoffigures
\listoftables
\thispagestyle{empty}
\clearpage
\setcounter{page}{1}

\section{Introduction}
\subsection{Context}
Nowadays, due to the availability of a huge variety of smart electronic devices, more and more applications are developed to help people in their day-to-day activities. In the healthcare field, wearable devices such as smartwatches are highly useful since they can be used to collect information about general well-being of users by means of mobile sensor technologies. As expected, measured data has several possible applications including, patient diagnostics and treatment or research motivations. \\

\textbf{TrackMe Technologies} is a company that develops health-monitoring devices devoted to measure and record different parameters related to the health status of a person (i.e. body temperature, blood pressure, heart pulse rate and percentage of O2 in the blood) and also their location. TrackMe health smartwatch is synchronized with an app that gives users access to their data and stats. 

\subsection{Purpose}
Taking into account the long list of currently available wearable devices, \textbf{TrackMe} is continuously looking for new strategic decisions to combat competition by offering new innovative services. In this opportunity, they decided to generate revenues from user data in a direct way (i.e. extend its business model by implementing \textbf{data trading}). This is, selling collected data to third parties -who need to know the health status of the population for different purposes- in an anonymised form.\\

TrackMe new software-based service is called \textbf{Data4Help}. This service provides registered third-party companies the possibility to monitor location and body metrics of individuals by exploiting data acquired through their wearable devices.\\

After some time, TrackMe realizes that a good part of its third-party customers wants to use the data acquired through Data4Help to offer a personalized SOS service to elderly people and decides to  build a new service, called \textbf{AutomatedSOS}, on top of Data4Help. AutomatedSOS provides a personal alarm service for the elderly subscribed customers by monitoring their health status.\\

Finally, TrackMe realizes that another great source of revenues could be the development of a service to track athletes participating in a run. In this case, the service, called \textbf{Track4Run}, will allow run organizers to define the path, TrackMe wearable-devices users to enroll, and spectators to see on the map the position of all runners during the run. 

\subsection{Scope}

\subsubsection{Description of the given problem}
\textbf{TrackMe} offers to their customers a new service called \textbf{Data4Help}, when they agree provides the user, an interface for the registration of individuals who, by registering, agree that TrackMe acquires their data. They are wirelessly connected to each other, and all that needs to be done is to view them either using a simple PC or a smart-phone. Also, it supports the registration of third parties, who can thereafter request for the data of some specific individuals (using SSN) and the request will be sent to the respective individuals (who can Accept or Reject it) or can ask for bulk data based on some filtering criteria provided by the system (such as age, gender, country, province) which will be handled directly by TrackMe (Requests accepted only if, for which the number of individuals whose data satisfy the request is higher than 1000). As the request for data acquisition is approved, TrackMe offers third party customers to subscribe to the new data (if they want to), in order to receive new data as soon as it is produced.\\

\textbf{AutomatedSOS} is built on the top of Data4Help, which gives an opportunity to the people above the age of 50 years, if they want to subscribe to this new SOS service. AutomatedSOS monitors the health status of the subscribed customers and, when vital signs parameters (Heart rate, Blood pressure) are below certain thresholds, sends to the location of the customer an ambulance, guaranteeing a reaction time of less than 5 seconds from the time the parameters	are	below the threshold.\\

\textbf{Track4Run} allows the 'fit' customers to participate in the upcoming run (by sending a request), if they accept it, they will be redirected to the page where, they can enrol themselves to the run.\\

\subsubsection{World and Machine Phenomenon}
\begin{itemize}
\item \textbf{World Phenomenon}
\\
\\
In TrackMe system, the portions of the world which machine cannot observe are classified as:
\begin{enumerate}
\item{} Wearable devices (Smartwatches or similar devices)
\item{} Individuals (who are sharing their personal data after successful registration)
\item{} Third party customers (who registers in TrackMe and are willing to use the data acquired using Data4Help)
\item{} Ambulance 
\end{enumerate}

\item \textbf{Shared Phenomenon}
\\
\\
The Shared Phenomenon between the real world and TrackMe (Data4Help, AutomatedSOS and Track4Run) is classified as:
\begin{enumerate}
\item{} The data collected by the devices (i.e. Blood pressure, Body temperature, Heart rate)
\item{} Individuals location and health status
\item{} Ambulance contact numbers (emergency number)
\item{} Defined path for the run
\item{} The current location of the athletes participating during the run
\end{enumerate}
\end{itemize}

\subsubsection{Goals}
\begin{itemize}
\item{\textbf{Data4Help}}\\
$[G1]$ Provide a service capable to store the location and physical data of an individual obtained by means of TrackMe's smart devices\\
$[G2]$ Provide a service that let a third party company to access the stored data from an individual\\
$[G3]$ Provide a service that let a third party company to access anonymized stored data of groups of individuals subject to specified constraints\\
$[G4]$ Provide to a third party company a way to get updates on a specific individual's data or of a previously saved search of anonymized data\\
 \item{\textbf{AutomatedSOS}}\\
$[G5]$ Provide a service capable to send an advice to the health care service when a individual's parameters are below a defined threshold\\
 \item{\textbf{Track4Run}}\\
$[G6]$ Provide a platform that let races organizer to define race's path, and let the participants to enroll to them\\
$[G7]$ Provide to run spectators a way to watch the participants' location during the race\\
\end{itemize}

\subsection{Definitions, Acronyms, Abbreviations}
\subsubsection{Definitions}
\begin{itemize}
\item{\textbf{Data trading}}: Generate revenew from user data in a much more direct way, by selling user data to a third party.
\item{\textbf{Health status}}: Collection of the last measured overall physical health parameters of a user or a group of users.
\item{\textbf{Remote monitoring}}: Remote Monitoring (RMON) is a standard specification that facilitates the monitoring of network operational activities through the use of remote devices known as monitors or probes(here, we are using smartwatches).
\item{\textbf{Wearable device}}: Devices that can be used to collect data and monitor users' overall physical health, such as body temperature, blood pressure, heart pulse rate, etc.

\end{itemize}
\subsubsection{Acronyms}
\begin{itemize}
\item{RASD}: Requirement Analysis and Specification Document

\end{itemize}

\subsubsection{Abbreviations}
\begin{itemize}
\item $[Gn]$: n-goal. 
\item $[Dn]$: n-domain assumption. 
\item $[Rn]$: n-functional requirement. 

\end{itemize}

\subsection{Revision History}

  \begin{tabular}{ | l | c |}
    \hline
    \textbf{Version} & \textbf{Last modified date} \\ \hline
    1.0 & \nth{11} November, 2018 \\ \hline
  \end{tabular}


\subsection{Reference Documents}
\begin{itemize}
\item Requirement Analysis and Specification Document: AA 2017-2018.pdf”. Version 1.0 - 26.10.2017
\item Henriksen, A., Haugen Mikalsen, M., Woldaregay, A. Z., Muzny, M., Hartvigsen, G., Hopstock, L. A., Grimsgaard, S. (2018)
\\Using Fitness Trackers and Smartwatches to Measure Physical Activity in Research: Analysis of Consumer Wrist-Worn Wearables. Journal of medical Internet research, 20(3), e110. doi:10.2196/jmir.9157. 
\\Retrieved from: https://www.ncbi.nlm.nih.gov/pmc/articles/PMC5887043/
\item IEEE. (1993). IEEE Recommended Practice for Software Requirements Specifications (IEEE 830-1993). 
\\Retrieved from https://standards.ieee.org/standard/830-1993.html


\end{itemize}
\subsection{Document Structure}
This document is divided in six parts, each one devoted to approach each one of the steps required to apply requirements engineering techniques.\\
\begin{itemize}
\item Chapter 1 gives an introduction to the problem and describes the purpose of the application TrackMe. The scope of the application is defined by stating the goals and description of the problem.
\item Chapter 2 presents the overall description of the project. The product perspective includes details on the shared phenomena and the domain models.
\item Chapter 3 contains the external interface requirements, including: user interfaces, hardware interfaces, software interfaces and communication interfaces. Furthermore, the functional requirements are defined by using use case and sequence diagram. The non-functional requirements are defined through performance requirements, design constraints and software system attributes.
\item Chapter 4 includes the alloy model and the discussion of its purpose. Also, a world generated by it is shown.
\item Chapter 5 shows the effort spent by each group member while working on this project.
\item Chapter 6 includes the reference documents.

\end{itemize}

\section{Overall Description}
\subsection{Product Perspective}

\end{document}