% PACKAGES
\documentclass[a4paper, hidelinks, 12pt]{report}
\usepackage[margin=1in]{geometry}
\usepackage{amsfonts,amsmath,amssymb}
\usepackage[none]{hyphenat}
\usepackage{fancyhdr}
\usepackage{graphicx}
\usepackage{float}
\usepackage[nottoc,notlot,notlof]{tocbibind}
\usepackage{hyperref}
\usepackage{longtable}
\usepackage[utf8]{inputenc}
\usepackage{booktabs}
\usepackage{multirow}
\usepackage{booktabs}
\usepackage[font=footnotesize]{caption}
\usepackage[flushleft]{threeparttable}
\usepackage{amsmath}
\usepackage{relsize}
\usepackage[super,negative]{nth}
\usepackage{enumerate}
\usepackage{float}
\usepackage[counterclockwise, figuresleft]{rotating}
%%%%%%%%%%%%

% DOC STYLES
\makeatletter
\def\thickhrulefill{\leavevmode \leaders \hrule height 1ex \hfill \kern \z@}
\def\@makechapterhead#1{
	\vspace*{4\p@}
	{\parindent \z@ \centering \reset@font
		\thickhrulefill\quad
		\scshape \@chapapp{} \thechapter
		\quad \thickhrulefill
		\par\nobreak
		\vspace*{4\p@}
		\interlinepenalty\@M
		\hrule
		\vspace*{4\p@}
		\Huge \bfseries #1\par\nobreak
		\par
		\vspace*{4\p@}
		\hrule
		\vskip 50\p@
}}
\def\@makeschapterhead#1{
	\vspace*{4\p@}
	{\parindent \z@ \centering \reset@font
		\thickhrulefill
		\par\nobreak
		\vspace*{4\p@}
		\interlinepenalty\@M
		\hrule
		\vspace*{4\p@}
		\Huge \bfseries #1\par\nobreak
		\par
		\vspace*{4\p@}
		\hrule
		\vskip 50\p@
}}

\pagestyle{fancy}
\fancyhead{}
\fancyfoot{}
\fancyhead[L]{\slshape\MakeUppercase{\textbf{DD}}}
\fancyhead[R]{\slshape{Avila, Schiatti, Virdi}}
\fancyfoot[C]{\thepage}
\renewcommand{\footrulewidth}{1pt}
\renewcommand{\headrulewidth}{1pt}
\linespread{1.3}
%\floatstyle{boxed}
\restylefloat{figure}
\parindent 0ex
%\renewcommand{baselinestretch}{1.5}

%%%%%%%%%%%%

% COMMANDS
\newcommand\requirement[1]{\item[{[R#1]}] }
\newcommand\goal[1]{\item[{[G#1]}] }
\newcommand\assumption[1]{\item[{[D#1]}] }
\newcommand\usecase[1]{ [UC#1] }

%%%%%%%%%%%%

%BODY
\begin{document}
	\begin{titlepage}
		\centering
		\vspace*{0.7 cm}
		\includegraphics[scale = 0.85]{../Assets/PolimiLogo.png}\\[1.6 cm]
		\textsc{\large Department of Computer Science and Engineering}\\[1.8 cm]
		
		\rule{\linewidth}{0.2 mm} \\[0.4 cm]
		{ \huge \bfseries Design Document (DD)}\\
		\rule{\linewidth}{0.2 mm} \\[1.5 cm]
		
		\textsc{\Large TrackMe}\\[0.5 cm]
		\textsc{\large - v1.0 -}\\[1 cm]
		
		\begin{minipage}{0.4\textwidth}
			\begin{flushleft} \large
				\emph{Authors:}\\
				\textbf{Avila}, Diego \\
				\textbf{Schiatti}, Laura \\
				\textbf{Virdi}, Sukhpreet
			\end{flushleft}
		\end{minipage}~
		\begin{minipage}{0.4\textwidth}
			\begin{flushright} \large
				%\emph{Student Number:} \\
				903988 \\
				904738 \\
				904204
			\end{flushright}
		\end{minipage}\\[2 cm]
		
		{\large December \nth{10} , 2018}\\[2 cm]
		
		\vfill
	\end{titlepage}
	
	\pagenumbering{roman}
	\tableofcontents
%	\thispagestyle{empty}
	\newpage
	\listoffigures
	\listoftables
%	\thispagestyle{empty}
	\clearpage
	\pagenumbering{arabic}
	\setcounter{page}{1}
	
	\chapter{Introduction}
	\section{Context}
	
	\section{Purpose}
		
	\section{Scope}
	
	\section{Definitions, Acronyms, Abbreviations}
	\subsection{Definitions}
	\begin{itemize}
		\item{\textbf{Data trading}}: Generate revenue from user data in a much more direct way, by selling user data to a third party.
	\end{itemize}
	
	\subsection{Acronyms}
	\begin{itemize}
		\item{DD}: Design Document
		\item{D4H}: Data4Help
		\item{ASOS}: AutomatedSOS
		\item{T4R}: Track4Run
	\end{itemize}
	
	\subsection{Abbreviations}
	\begin{itemize}
		\item $[Gn]$: n-goal.
	\end{itemize}
	
	\section{Revision history}
	It is important to keep track of the revisions made to this document: \\
	
	\begin{table}[h]
		\centering
		\begin{tabular}{c c}
			\hline\hline
			\textbf{Version} & \textbf{Last modified date} \\ [0.5ex]
			\hline
			1.0 &  \nth{10} December, 2018  \\
			\hline
		\end{tabular}
		\caption{Revision history timeline}
		\label{fig:Revision history}
	\end{table}
	
	\section{Document structure}
	This document is divided in six parts, each one devoted to approach each one of the steps required to apply requirements engineering techniques.
	\begin{itemize}
		\item Chapter 1 gives ...
		\item Chapter 2 presents ...
		\item Chapter 3 ...
		\item Chapter 4 includes ....
		\item Chapter 5 shows the effort spent by each group member while working on this project.
		\item Chapter 6 includes the reference documents.
	\end{itemize}
	
	\chapter{Architectural design}
	High-level components and their interaction
	
	\section{Overview}
	\section{Component view}
		\subsection{Data4Help component diagram}
		In the Figure \ref{fig:d4h_component_diagram} it can be seen the component diagram of D4H, with all its external interfaces. It can be noticed two packages: \textit{Data Base} and \textit{D4H Back-end}, where the former one refers to the TrackMe database, and which provides an interface used by the \textbf{DBManager} component. \\\\
		On the other hand, \textit{D4H Back-end} package, contains all the components related to D4H, which are needed to provide the interfaces used by the third parties and the web site. The following interfaces are used by the web site: \textbf{SignupWeb}, \textbf{LoginWeb}, \textbf{SearchWeb}, \textbf{RequestWeb} and  \textbf{SubscriptionWeb}, while the \textbf{RequestAPI} interface is used by the third parties. In this case, D4H provides the interface in order to let the third parties send Requests for accessing the health status and location of the individuals to them.\\\\
		Furthermore, \textbf{AuthenticatorService} component has the responsibility of validate the different credentials, and to provide the Secret codes to the third parties.\\\\
		Finally, it can be seen the different third parties related to D4H. It worth mentioning \textbf{Track4Run} and \textbf{AutomatedSOS} components which, even though they are part of the TrackMe environment, they are treated as third parties in the sense they are completely decoupled of D4H. Moreover, all third parties must provide a port to D4H in order to let it communicate the incoming changes of the subscriptions.
		
			\begin{sidewaysfigure}
    				\centering
				\includegraphics[width=1\textwidth]{diagrams/d4h_component_diagram.png}
				\caption[Data4Help Component Diagram]{Data4Help Component Diagram}
				\label{fig:d4h_component_diagram}
			\end{sidewaysfigure}	
			\clearpage
			
		\subsection{Track4Run component diagram}	
			In the Figure \ref{fig:t4r_component_diagram} it can be seen the T4R components. Unlike the D4H component diagram, here the \textit{Database} package is inside the \textit{T4R Back-end} package, this is so because T4R owns it in the sense that it has full responsibility on it. Besides this detail, the database provides a connector in order to let the \textbf{DBManager} component communicate to it.\\\\
			The main structure of the system is similar to the structure seen in D4H component diagram. The web site communicates with the back-end using the following interfaces: \textbf{LoginWeb}, \textbf{SignupWeb}, \textbf{UserWeb} and \textbf{NotificationWeb}. On the other hand, T4R provides the \textbf{DataAPI} interface, which is responsibly of receiving all the data of the participants, and uses the \textbf{RequestAPI} interface, in order to send the requests for accessing the individuals' location and health status.\\\\
			Finally, as in D4H, the \textbf{AuthenticatorService} component is responsible of validate the users' credentials.
			\begin{figure}[H]
				\centering
				\includegraphics[width=1\textwidth]{diagrams/t4r_component_diagram.png}
				\caption[Track4Run Component Diagram]{Track4Run Component Diagram}
				\label{fig:t4r_component_diagram}
			\end{figure}	
			
			\subsection{AutomatedSOS component diagram}	
			In the Figure \ref{fig:asos_component_diagram} 
			\begin{figure}[H]
				\centering
				\includegraphics[width=1\textwidth]{diagrams/asos_component_diagram.png}
				\caption[AutomatedSOS Component Diagram]{AutomatedSOS Component Diagram}
				\label{fig:asos_component_diagram}
			\end{figure}	
			
			
	\section{Deployment view}
	
	\section{Runtime view}
	You can use sequence diagrams to describe the way components
interact to accomplish specific tasks typically related to your use cases

	\section{Selected architectural styles and patterns}
	Please explain which styles/patterns you
used, why, and how 

	\section{Other design decisions}
	
	\chapter{User interface design}
	Provide an overview on how the user interface(s) of your system will look like; if you have included this part in the RASD, you can simply refer to what you have already done, possibly, providing here some extensions if applicable.
	
	\chapter{Requirements traceability}
	Explain how the requirements you have defined in the RASD map to the design elements that you have defined in this document.
	
	\chapter{Implementation, integration and test plan}
	Identify here the order in which you plan to implement the subcomponents of your system and the order in which you plan to integrate such subcomponents and test the integration.
	
	\chapter{Effort spent}
	\begin{table}[h]
		\centering
		\begin{tabular}{l c}
			\hline\hline
			\multicolumn{2}{c}{\textbf{Team Work}} \\
			\hline
			\textbf{Task} & \textbf{Hours} \\ [0.5ex]
			\hline
			Understanding the problem & 3  \\
			Brainstorming & 2 \\
			World and shared phenomena & 2 \\
			Definitions, acronyms, abbreviations & 1  \\
			Software system attributes & 2 \\
			Alloy coding & 7 \\
			Checking document  & 4  \\
			\hline
			\textbf{Total} & 20  \\
			\hline
		\end{tabular}
		\caption{Time spent by all team members}
		\label{fig:Time spent by all team members}
	\end{table}
	
	\begin{table}[h]
		\centering
		\begin{tabular}{l c l c l c}
			\hline\hline
			\multicolumn{6}{c}{\textbf{Individual Work}} \\
			\hline
			\multicolumn{2}{c |}{\textbf{Diego Avila}}  &
			\multicolumn{2}{c |}{\textbf{Laura Schiatti}} &
			\multicolumn{2}{c}{\textbf{Sukhpreet Kaur}}  \\
			\hline
			\textbf{Task} & \textbf{Hours}
			& \textbf{Task} & \textbf{Hours}
			& \textbf{Task} & \textbf{Hours} \\ [0.5ex]
			\hline
			X &  X
			& Layout & X
			& X & X  \\
			\hline
			\textbf{Total} & X
			& \textbf{Total} & X
			& \textbf{Total} & X  \\
			\hline
		\end{tabular}
		\caption{Time spent by each team member}
		\label{fig:Time spent by each team member}
	\end{table}
	
	\chapter{References}
	\begin{itemize}
		\item Requirement Analysis and Specification Document: AA 2017-2018.pdf”. Version 1.0 - 26.10.2017
		\item Henriksen, A., Haugen Mikalsen, M., Woldaregay, A. Z., Muzny, M., Hartvigsen, G., Hopstock, L. A., Grimsgaard, S. (2018)
		\\Using Fitness Trackers and Smartwatches to Measure Physical Activity in Research: Analysis of Consumer Wrist-Worn Wearables. Journal of medical Internet research, 20(3), e110. doi:10.2196/jmir.9157.
		\\Retrieved from: https://www.ncbi.nlm.nih.gov/pmc/articles/PMC5887043/
		\item IEEE. (1993). IEEE Recommended Practice for Software Requirements Specifications (IEEE 830-1993).
		\\Retrieved from https://standards.ieee.org/standard/830-1993.html
		\item Sloane, A. M. (2009). Software Abstractions: Logic, Language, and Analysis by Jackson Daniel, The MIT Press, 2006, 366pp, ISBN 978-0262101141.
	\end{itemize}
	
\end{document}
