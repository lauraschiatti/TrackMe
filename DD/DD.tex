% PACKAGES
\documentclass[a4paper, hidelinks, 12pt]{report}
\usepackage[margin=1in]{geometry}
\usepackage{amsfonts,amsmath,amssymb}
\usepackage[none]{hyphenat}
\usepackage{fancyhdr}
\usepackage{graphicx}
\usepackage{float}
\usepackage[nottoc,notlot,notlof]{tocbibind}
\usepackage{hyperref}
\usepackage{longtable}
\usepackage[utf8]{inputenc}
\usepackage{booktabs}
\usepackage{multirow}
\usepackage{booktabs}
\usepackage[font=footnotesize]{caption}
\usepackage[flushleft]{threeparttable}
\usepackage{amsmath}
\usepackage{relsize}
\usepackage[super,negative]{nth}
\usepackage{enumerate}
\usepackage{float}
\usepackage[counterclockwise, figuresleft]{rotating}
\usepackage[dvipsnames]{xcolor}
\usepackage{listings}
%%%%%%%%%%%%

% DOC STYLES
\makeatletter
\def\thickhrulefill{\leavevmode \leaders \hrule height 1ex \hfill \kern \z@}
\def\@makechapterhead#1{
	\vspace*{4\p@}
	{\parindent \z@ \centering \reset@font
		\thickhrulefill\quad
		\scshape \@chapapp{} \thechapter
		\quad \thickhrulefill
		\par\nobreak
		\vspace*{4\p@}
		\interlinepenalty\@M
		\hrule
		\vspace*{4\p@}
		\Huge \bfseries #1\par\nobreak
		\par
		\vspace*{4\p@}
		\hrule
		\vskip 50\p@
}}
\def\@makeschapterhead#1{
	\vspace*{4\p@}
	{\parindent \z@ \centering \reset@font
		\thickhrulefill
		\par\nobreak
		\vspace*{4\p@}
		\interlinepenalty\@M
		\hrule
		\vspace*{4\p@}
		\Huge \bfseries #1\par\nobreak
		\par
		\vspace*{4\p@}
		\hrule
		\vskip 50\p@
}}

\pagestyle{fancy}
\fancyhead{}
\fancyfoot{}
\fancyhead[L]{\slshape\MakeUppercase{\textbf{DD}}}
\fancyhead[R]{\slshape{Avila, Schiatti, Virdi}}
\fancyfoot[C]{\thepage}
\renewcommand{\footrulewidth}{1pt}
\renewcommand{\headrulewidth}{1pt}
\linespread{1.3}
%\floatstyle{boxed}
\restylefloat{figure}
\parindent 0ex
%\renewcommand{baselinestretch}{1.5}

%%%%%%%%%%%%

% COMMANDS
\newcommand\requirement[1]{\item[{[R#1]}] }
\newcommand\goal[1]{\item[{[G#1]}] }
\newcommand\assumption[1]{\item[{[D#1]}] }
\newcommand\usecase[1]{ [UC#1] }

%%%%%%%%%%%%

%BODY
\begin{document}
	\begin{titlepage}
		\centering
		\vspace*{0.7 cm}
		\includegraphics[scale = 0.85]{../Assets/PolimiLogo.png}\\[1.6 cm]
		\textsc{\large Department of Computer Science and Engineering}\\[1.8 cm]
		
		\rule{\linewidth}{0.2 mm} \\[0.4 cm]
		{ \huge \bfseries Design Document (DD)}\\
		\rule{\linewidth}{0.2 mm} \\[1.5 cm]
		
		\textsc{\Large TrackMe}\\[0.5 cm]
		\textsc{\large - v1.0 -}\\[1 cm]
		
		\begin{minipage}{0.4\textwidth}
			\begin{flushleft} \large
				\emph{Authors:}\\
				\textbf{Avila}, Diego \\
				\textbf{Schiatti}, Laura \\
				\textbf{Virdi}, Sukhpreet
			\end{flushleft}
		\end{minipage}~
		\begin{minipage}{0.4\textwidth}
			\begin{flushright} \large
				%\emph{Student Number:} \\
				903988 \\
				904738 \\
				904204
			\end{flushright}
		\end{minipage}\\[2 cm]
		
		{\large December \nth{10} , 2018}\\[2 cm]
		
		\vfill
	\end{titlepage}
	
	\pagenumbering{roman}
	\tableofcontents
%	\thispagestyle{empty}
	\newpage
	\listoffigures
	\listoftables
%	\thispagestyle{empty}
	\clearpage
	\pagenumbering{arabic}
	\setcounter{page}{1}
	
	\chapter{Introduction}
	\section{Context}
	\textbf{TrackMe} develops health-monitoring devices devoted to measure and record different parameters related to the health status of a person (i.e. body temperature, blood pressure, heart pulse rate and percentage of O2 in the blood) and also their location. TrackMe health smartwatch is synchronized with an app that gives users access to their data and stats. This document discuss more technical aspects of working of the TrackMe application.
	
	\section{Purpose}
	The purpose of this document is to give more technical details than the RASD about TrackMe application system. While the RASD presented a general view of the system and what functions the system is supposed to execute, this document aims to present the implementation of the system including components, run-time processes, deployment and algorithm design. It also presents in more details the implementation and integration plan, as well as the testing plan.\\\\
More precisely, the document presents:
\begin{itemize}
		\item{}Overview of the high level architecture
		\item{}The main components and their interfaces provided one for another
		\item{}The Runtime behaviour
		\item{}The design patterns
		\item{}The algorithm design of the most critical parts of the application
		\item{}Implementation plan
		\item{}Integration plan
		\item{}Testing Plan
	\end{itemize}
The purpose of this document is to provide an overall guidance to the architecture of the software product.
		
	\section{Scope}
	The TrackMe environment will be composed by three systems, whose scope are defined in this section. \\
	
	\textbf{Data4Help} is a system capable to store the physical health data of any individual using any TrackMe wearable device. All the stored data will be available to any third party company, after request and approval of its use is sent to the TrackMe wearable device user, who will be able to accept or reject the request, under no condition. Also, third party companies will be able to request data of a group of users, and Data4Help will provide that information under the solely condition of be able to anonymize it.\\
	
	In addition, \textbf{AutomatedSOS} is a system that will be built on top of Data4Help, and will send requests to every elderly individual and responds back, if accepted, it adds to ASOS database in order to access its physical health parameters. ASOS is able to respond only if a city is available for health-care service in that city, before creating a request to that user. Apparently, ASOS automatically assigns emergency contact according to the city of the user. It will monitor the physical health status of all of its users in order to establish their health condition, based on previously defined thresholds. If any of the user's health parameters is under or above its threshold, the system will contact the associated health-care service for that user. AutomatedSOS will do its best effort to contact the health-care service, but under no circumstances will follow-up the status of the health-care response, meaning that it will not know if the health-care service answered the notification or not. \\
	
	Finally, \textbf{Track4Run}, also a system built on top of Data4Help, will send requests to every user in order to access their location and let them participate in a defined running circuit. Track4Run organizer users, will be able to setup a race, define its running circuit and send invitations to all the TrackMe wearable device users. Also, the race spectators will be able to follow the participants' location using the Track4Run web site. The location of every participant will only be available during the race.
	
	\section{Definitions, Acronyms, Abbreviations}
	\subsection{Definitions}
	\begin{itemize}
		\item{\textbf{Data trading}}: Generate revenue from user data in a much more direct way, by selling user data to a third party.
	\end{itemize}
	
	\subsection{Acronyms}
	\begin{itemize}
		\item{DD}: Design Document
		\item{D4H}: Data4Help
		\item{ASOS}: AutomatedSOS
		\item{T4R}: Track4Run
		\item{GUI}: Graphical User Interface
		\item{MVC}: Model View Controller is a design pattern used for GUIs
		\item{RASD}: Requirements Analysis and Specifications Document
	\end{itemize}
	
	\subsection{Abbreviations}
	\begin{itemize}
		\item $[Gn]$: n-goal.
	\end{itemize}
	
	\section{Revision history}
	It is important to keep track of the revisions made to this document: \\
	
	\begin{table}[h]
		\centering
		\begin{tabular}{c c}
			\hline\hline
			\textbf{Version} & \textbf{Last modified date} \\ [0.5ex]
			\hline
			1.0 &  \nth{10} December, 2018  \\
			\hline
		\end{tabular}
		\caption{Revision history timeline}
		\label{fig:Revision history}
	\end{table}
	
	\section{Document structure}
	This document is divided in seven parts, each one devoted to approach each one of the steps required to apply requirements engineering techniques.
	\begin{itemize}
		\item Chapter 1 gives an introduction of the design document. It contains the purpose and the scope of the document, as well as some abbreviation in order to provide a better understanding of the document to the reader.
		\item Chapter 2  deals with the architectural design of the application. It gives an overview of the architecture
and it also contains the most relevant architecture views: component view, class view, deployment view, runtime view and it shows the interaction of the component interfaces. Some of the used architectural designs and designs patterns are also presented here, with an explanation of each one of them and the purpose of their usage.

		\item Chapter 3  refers to the mock-ups already presented in the RASD document.
		\item Chapter 4 explains how the requirements that have been defined in the RASD map to the design elements that are defined in this document.
		\item Chapter 5 presents the algorithm design. It includes the most critical parts of the application and the
algorithms designed to deal with them. All of the algorithms are written in pseudo code in order not to make any restriction on the implementation language and stay focused on the most important parts.
		\item Chapter 6 shows the effort spent by each group member while working on this project.
		\item Chapter 7 includes the reference documents.
	\end{itemize}
	
	\chapter{Architectural design}
	
	\section{Overview}
	Application architecture design is a process which has to be executed in a defined flow. High-level components and their interaction is displayed as follows:
	\begin{figure}[H]
				\centering
				\includegraphics[width=1\textwidth]{diagrams/architecture.png}
				\caption[High-level architecture Diagram]{High-level architecture Diagram}
				\label{fig:High-level architecture Diagram}
			\end{figure}
			
The above architecture diagram depicts clearly that third party - Data4Help relation is a service oriented, when the third-party user asks for some specific data and it is an event based, when it is subscribed to the data. In addition, a CRON job runs every day at 00:00, if there is every user above age of 60 years - they receive an ASOS notification. The system uses MongoDB, that listens, watch collection and events. Apparently, all the APIs are RESTful.\\
When the third-party users have subscribed to the bulk data subscription, they define an update time when they want to get the refreshed data (e.g. update data in every 'X' hours).

	\section{Component view}
		\subsection{Data4Help component diagram}
		In the Figure \ref{fig:d4h_component_diagram} it can be seen the component diagram of D4H, with all its external interfaces. It can be noticed two main components: \textbf{DataBase} and \textbf{D4HBackend}, where the former one refers to the TrackMe database, and which provides an interface used by the \textbf{DBManager} component. \\\\
		On the other hand, \textbf{D4HBackend} component, contains all the components related to D4H, which are needed to provide the interfaces used by the third parties and the web site. The following interfaces are used by the web site: \textbf{SignupWeb}, \textbf{LoginWeb}, \textbf{SearchWeb}, \textbf{RequestWeb} and  \textbf{SubscriptionWeb}, while the \textbf{RequestAPI} interface is used by the third parties. In this case, D4H provides the interface in order to let the third parties send requests for accessing the health status and location of the individuals to them.\\\\
		Furthermore, \textbf{AuthenticatorService} component has the responsibility of validate the different credentials, and to provide the secret codes to the third parties, to do so, it interacts with the \textbf{TokenDB} component using the \textbf{TokenConnector}\\\\
		Finally, it can be seen the different third parties related to D4H. It worth mentioning \textbf{Track4Run} and \textbf{AutomatedSOS} components which, even though they are part of the TrackMe environment, they are treated as third parties in the sense they are completely decoupled of D4H. Moreover, all third parties must provide an interface to D4H in order to let it communicate the incoming changes of the subscriptions.
		
			\begin{sidewaysfigure}
    				\centering
				\includegraphics[width=1\textwidth]{diagrams/d4h_component_diagram.png}
				\caption[Data4Help Component Diagram]{Data4Help Component Diagram}
				\label{fig:d4h_component_diagram}
			\end{sidewaysfigure}	
			\clearpage
			
		\subsection{Track4Run component diagram}	
			In the Figure \ref{fig:t4r_component_diagram} it can be seen the T4R components. Unlike the D4H component diagram, here the \textit{DataBase} component is part of the \textit{T4RBackend} component, this is so because T4R owns it in the sense that it has full responsibility on it. Besides this detail, the database provides a connector in order to let the \textbf{DBManager} component communicate to it.\\\\
			The main structure of the system is similar to the structure seen in D4H component diagram. The web site communicates with the back-end using the following interfaces: \textbf{LoginWeb}, \textbf{SignupWeb}, \textbf{UserWeb} and \textbf{NotificationWeb}. On the other hand, T4R provides the \textbf{DataAPI} interface, which is responsibly of receiving all the data of the participants, and uses the \textbf{RequestAPI} interface, in order to send the requests for accessing the individuals' location and health status.\\\\
			Finally, as in D4H, the \textbf{AuthenticatorService} component is responsible of validate the users' credentials.
			\begin{figure}[H]
				\centering
				\includegraphics[width=1\textwidth]{diagrams/t4r_component_diagram.png}
				\caption[Track4Run Component Diagram]{Track4Run Component Diagram}
				\label{fig:t4r_component_diagram}
			\end{figure}	
			
			\subsection{AutomatedSOS component diagram}	
			In the Figure \ref{fig:asos_component_diagram} it can be seen the ASOS components. It can be noticed that the system has a similar structure of T4R: the \textbf{ASOSBackend} component owns the \textbf{DataBase} component, and provides a \textbf{DataAPI} interface to receive the updated information of the individuals.\\\\
			On the other hand, \textbf{ASOSBackend} component sends notifications to the different \textbf{Health Care Service} components using the provided \textbf{Alarm Interface}.
			\begin{figure}[H]
				\centering
				\includegraphics[width=1\textwidth]{diagrams/asos_component_diagram.png}
				\caption[AutomatedSOS Component Diagram]{AutomatedSOS Component Diagram}
				\label{fig:asos_component_diagram}
			\end{figure}	
			
			
	\section{Deployment view}
	
	\section{Runtime view}
	You can use sequence diagrams to describe the way components
interact to accomplish specific tasks typically related to your use cases

	\section{Selected architectural styles and patterns}
	Please explain which styles/patterns you
used, why, and how 

	\section{Other design decisions}
	
	\chapter{User interface design}
	The following mock-ups represent a basic idea of what the web application will look like after the first release:
	
	
	\begin{figure}[H]
		\centering
		\includegraphics[scale=0.4]{UI/d4h_home1.png}
		\includegraphics[scale=0.4]{UI/d4h_home2.png}
		\includegraphics[scale=0.4]{UI/d4h_home3.png}
		\includegraphics[scale=0.4]{UI/d4h_home4.png}
		\caption[UI: TrackMe's Home Page]{TrackMe's Home Page}
		\label{fig:Home_Page}
	\end{figure}
	
	As when new customers visits the home page of TrackMe, they can read about the work that our company does, what services we offer, what benefits users can achieve by joining the community. In addition, they can buy the wearables and get more information, how to use them, what is TrackMe, more details about Data4Help, ASOS and Track4Run is provided here.\\
	\begin{figure}[H]
		\centering
		\includegraphics[scale=0.4]{UI/d4h.png}
		\includegraphics[scale=0.4]{UI/d4h2.png}
		\includegraphics[scale=0.35]{UI/d4h_home4.png}

		\caption[UI: Data4Help Information Page]{Data4Help Information Page}
		\label{fig:Data4Help Information Page}
	\end{figure}
	\begin{figure}[H]
		\centering
		\includegraphics[scale=0.4]{UI/asos.png}

		\caption[UI: ASOS Information Page]{ASOS Information Page}
		\label{fig:ASOS Information Page}
	\end{figure}
	
	Next, comes the web page through which users can Login into the system (if already registered):
	
	\begin{figure}[H]
		\centering
		\includegraphics[scale=0.35]{UI/login1.png}
		\includegraphics[scale=0.35]{UI/login.png}
		\includegraphics[scale=0.35]{UI/d4h_home4.png}
		\caption[UI: Login Page]{Login Page}
		\label{fig:Login}
	\end{figure}
	
	 And if not already registered into the system, they can register themselves through the Register web-page. There are separate register forms for the Individual users and for third party users visible below:
	 \begin{figure}[H]
		\centering
		\includegraphics[scale=0.35]{UI/register1.png}
		\includegraphics[scale=0.35]{UI/register2.png}
		\includegraphics[scale=0.35]{UI/register3.png}
		\includegraphics[scale=0.35]{UI/d4h_home4.png}
		\caption[UI: Registration Page]{Registration Page}
		\label{fig:Registration}
	\end{figure}

	
	\chapter{Requirements traceability}
	Explain how the requirements you have defined in the RASD map to the design elements that you have defined in this document.
	
	\chapter{Implementation, integration and test plan}
	Identify here the order in which you plan to implement the subcomponents of your system and the order in which you plan to integrate such subcomponents and test the integration.
	
	\chapter{Effort spent}
	\begin{table}[h]
		\centering
		\begin{tabular}{l c}
			\hline\hline
			\multicolumn{2}{c}{\textbf{Team Work}} \\
			\hline
			\textbf{Task} & \textbf{Hours} \\ [0.5ex]
			\hline
			Understanding the problem & 3  \\
			Brainstorming & 2 \\
			World and shared phenomena & 2 \\
			Definitions, acronyms, abbreviations & 1  \\
			Software system attributes & 2 \\
			Alloy coding & 7 \\
			Checking document  & 4  \\
			\hline
			\textbf{Total} & 20  \\
			\hline
		\end{tabular}
		\caption{Time spent by all team members}
		\label{fig:Time spent by all team members}
	\end{table}
	
	\begin{table}[h]
		\centering
		\begin{tabular}{l c l c l c}
			\hline\hline
			\multicolumn{6}{c}{\textbf{Individual Work}} \\
			\hline
			\multicolumn{2}{c |}{\textbf{Diego Avila}}  &
			\multicolumn{2}{c |}{\textbf{Laura Schiatti}} &
			\multicolumn{2}{c}{\textbf{Sukhpreet Kaur}}  \\
			\hline
			\textbf{Task} & \textbf{Hours}
			& \textbf{Task} & \textbf{Hours}
			& \textbf{Task} & \textbf{Hours} \\ [0.5ex]
			\hline
			X &  X
			& Layout & X
			& X & X  \\
			\hline
			\textbf{Total} & X
			& \textbf{Total} & X
			& \textbf{Total} & X  \\
			\hline
		\end{tabular}
		\caption{Time spent by each team member}
		\label{fig:Time spent by each team member}
	\end{table}
	
	\chapter{References}
	\begin{itemize}
		\item Requirement Analysis and Specification Document: AA 2017-2018.pdf”. Version 1.0 - 26.10.2017
		\item Henriksen, A., Haugen Mikalsen, M., Woldaregay, A. Z., Muzny, M., Hartvigsen, G., Hopstock, L. A., Grimsgaard, S. (2018)
		\\Using Fitness Trackers and Smartwatches to Measure Physical Activity in Research: Analysis of Consumer Wrist-Worn Wearables. Journal of medical Internet research, 20(3), e110. doi:10.2196/jmir.9157.
		\\Retrieved from: https://www.ncbi.nlm.nih.gov/pmc/articles/PMC5887043/
		\item IEEE. (1993). IEEE Recommended Practice for Software Requirements Specifications (IEEE 830-1993).
		\\Retrieved from https://standards.ieee.org/standard/830-1993.html
		\item Sloane, A. M. (2009). Software Abstractions: Logic, Language, and Analysis by Jackson Daniel, The MIT Press, 2006, 366pp, ISBN 978-0262101141.
	\end{itemize}
	
\end{document}